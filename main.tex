\documentclass{article}
\usepackage[utf8]{inputenc}

\usepackage[blue]{gradschoolessentials}



\usepackage[colorlinks=true,
linkcolor=magenta,
urlcolor=green,
citecolor=blue]{hyperref}

\usepackage{bigfoot}
\usepackage{listings}
\lstset
{
    language={[LaTeX]TeX},
    basicstyle=\tt,
}

\usepackage{tcolorbox}
\newtcolorbox{mybox}[1]{colback=gray!5!white,colframe=gray!75!black,fonttitle=\bfseries,title=#1}

\newcommand{\verbprint}[1]{
\begin{minipage}{0.5\linewidth}
%\begin{verbatim}
    %\lstinline|#1|
%\end{verbatim}
\end{minipage}
\begin{minipage}{0.5\linewidth}
    $#1$
\end{minipage}
}

\newcommand{\verbsec}[1]{\subsection{\texttt{#1}}}
\newcommand{\verbsubsec}[1]{\subsubsection{\texttt{#1}}}


\begin{document}

\begin{center}
	\bigskip\bigskip
	%\vfill
	\huge{\sc $\mathsf{GradSchoolEssentials}$\\}
	\bigskip\bigskip\bigskip
	%\vfill
	\large{\LaTeX\, \it{Style Guide \---- ver. 0.9\\}}
    \bigskip
    \copyright 2021\---2023, Greg Riggs\\
    %\bigskip
    email: \url{rigzridge@gmail.com}
    %[WVU Dept. of Physics \& Astronomy]
\end{center}

\tableofcontents

\section{Motivation}

Hey there, \LaTeX\, ninja, and thanks for your interest in this project! It all started in 2021, when I finally tired of typing out some opaque, monstrous expression\footnote{ \verb|\frac{\partial\hat\boldsymbol{x}}{\partial t}| } when all I wanted to see was a simple $\pd{\bhats{x}}{t}$! With $\mathsf{gradschoolessentials}$, printing this expression is as easy as \verb|\pd{\bhats{x}}{t}|, which I believe is a more literate approach.
\\\\
Stated plainly, this style file provides macros to quickly and conveniently produce the high-quality typesetting that \TeX\, is famous for. The style guide itself is intended as a working document, so check the github (\url{https://github.com/rigzridge/gradschoolessentials}) for more information.

\breaknote

\section{Initialization}
Using $\mathsf{gradschoolessentials}$ is as simple as including as single \verb|\usepackage{}| command! The following is a minimum working example: 
\begin{lstlisting}
\documentclass{article}
\usepackage[utf8]{inputenc}

\usepackage[blue]{gradschoolessentials}

\begin{document}
$$\N\in\Z\in\Q\in\R\in\C$$
\end{document}
\end{lstlisting}

\section{Environments}
The $\mathsf{gradschoolessentials}$ style provides three new environments to simplify homework. Indexing is done automatically, and the operation of \verb|\label{}| and \verb|\ref{}| is analogous to its use in \verb|\section{}|, \verb|\subsection{}|, \verb|\subsubsection{}|.

\verbsec{problem}
Likely a homework problem, and numerically indexed. Importantly, the counter associated with \verb|problem| ({\it i.e.}, \verb|\theproblem|) will reset {when a new section is created}.
\begin{mybox}{Example of \texttt{problem} environment}
\begin{lstlisting}
\begin{problem}\label{prob::e^-tx}
    Here is an example problem. Show that 
    $$\int_0^\infty e^{-tx} dx = \frac{1}{t}.$$
\end{problem}
\end{lstlisting}
\begin{problem}\label{prob::e^-tx}
    Here is an example problem. Show that 
    $$\int_0^\infty e^{-tx} dx = \frac{1}{t}.$$
\end{problem}
\end{mybox}

\verbsec{subprob}
An alphabetically indexed subproblem. Will appear as current \verb|\sectioncolor| (must adjust \verb|gradschooolessentials.sty| to change).

\begin{mybox}{A \texttt{problem} with \texttt{subprob}}
\begin{lstlisting}
\begin{problem}\label{prob::factorial}
    \begin{subprob}\label{subprob::gamma}
        Using the results of Problem \ref{prob::e^-tx}, 
        verify the well-known relation 
        $$\int_0^\infty x^ne^{-x} dx = n!.$$
    \end{subprob}
    \begin{subprob}
        Confirm your result for part \ref{subprob::gamma} 
        using repeated integration by parts.
    \end{subprob}
\end{problem}
\end{lstlisting}
\begin{problem}\label{prob::factorial}
    \begin{subprob}\label{subprob::gamma}
        Using the results of Problem \ref{prob::e^-tx}, directly verify the well-known relation 
        $$\int_0^\infty x^ne^{-x} dx = n!.$$
    \end{subprob}
    \begin{subprob}
        Confirm your result for part \ref{subprob::gamma} using repeated integration by parts.
    \end{subprob}
\end{problem}
\end{mybox}

\verbsec{subsubprob}
A sub-subproblem, indexed by an italic, lower-case roman numeral. Will appear as current \verb|\sectioncolor|.

\begin{mybox}{A \texttt{problem} with \texttt{subprob} and \texttt{subsubprob}}
\begin{lstlisting}
\begin{problem}\label{prob::factorial}
    \begin{subprob}\label{subprob::gamma}
        Using the results of Problem \ref{prob::e^-tx}, 
        verify the well-known relation 
        $$\int_0^\infty x^ne^{-x} dx = n!.$$
    \end{subprob}
    \begin{subprob}
        Confirm your result for part \ref{subprob::gamma} 
        using repeated integration by parts.
    \end{subprob}
\end{problem}
\end{lstlisting}
\begin{problem}\label{prob::factorial}
    \begin{subprob}\label{subprob::gamma}
        Using the results of Problem \ref{prob::e^-tx}, verify the well-known relation 
        $$\int_0^\infty x^ne^{-x} dx = n!.$$
    \end{subprob}
    \begin{subprob}
        Confirm your result for part \ref{subprob::gamma} using repeated integration by parts.
    \end{subprob}
\end{problem}
\end{mybox}

\section{Tools}
\verbsec{hwtitle}
\verbsec{ans}
\verbsec{anst}

\section{Sets}
\verbsec{N}
Natural numbers
\\\\
\begin{minipage}{0.5\linewidth}
%\color[rgb]{0.1,0.5,0.9}
    \verb|$n\in\N\implies n+1\in\N$|
\end{minipage}
\begin{minipage}{0.5\linewidth}
    $n\in\N\implies n+1\in\N$
\end{minipage}

\verbsec{Z}
Integers
\\\\
\begin{minipage}{0.5\linewidth}
    \verb|$n\in\Z\implies n^2\in\N$|
\end{minipage}
\begin{minipage}{0.5\linewidth}
    $n\in\Z\implies n^2\in\N$
\end{minipage}

\verbsec{Q}
\verbsec{R}
\verbsec{C}
\verbsec{set}
\verbsubsec{bigset}
\verbsubsec{biggset}
\verbsubsec{Bigset}
\verbsubsec{Biggset}

\end{document}
